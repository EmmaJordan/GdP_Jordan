\documentclass[
11pt, % The default document font size, options: 10pt, 11pt, 12pt
% codirector, % Uncomment to add a codirector to the title page
]{charter} 

% El título de la memoria, se usa en la carátula y se puede usar el cualquier lugar del documento con el comando \ttitle
\titulo{Sistema de control de equipos de rayos X} 

% Nombre del posgrado, se usa en la carátula y se puede usar el cualquier lugar del documento con el comando \degreename
\posgrado{Carrera de Especialización en Sistemas Embebidos} 
%\posgrado{Carrera de Especialización en Internet de las Cosas} 
%\posgrado{Carrera de Especialización en Intelegencia Artificial}
%\posgrado{Maestría en Sistemas Embebidos} 
%\posgrado{Maestría en Internet de las cosas}

% Tu nombre, se puede usar el cualquier lugar del documento con el comando \authorname
\autor{Jordán Joan Emmanuel} 

% El nombre del director y co-director, se puede usar el cualquier lugar del documento con el comando \supname y \cosupname y \pertesupname y \pertecosupname
\director{Mgtr. Ing. Iriarte Eduardo}
\pertenenciaDirector{FING UNCuyo} 
\codirector{John Doe} % para que aparezca en la portada se debe descomentar la opción codirector en el documentclass
\pertenenciaCoDirector{FIUBA}

% Nombre del cliente, quien va a aprobar los resultados del proyecto, se puede usar con el comando \clientename y \empclientename
\cliente{Jordán Carlos Alberto}
\empresaCliente{SIMEN Rx}

% Nombre y pertenencia de los jurados, se pueden usar en cualquier lugar del documento con el comando \jurunoname, \jurdosname y \jurtresname y \perteunoname, \pertedosname y \pertetresname.
\juradoUno{Nombre y Apellido (1)}
\pertenenciaJurUno{pertenencia (1)} 
\juradoDos{Nombre y Apellido (2)}
\pertenenciaJurDos{pertenencia (2)}
\juradoTres{Nombre y Apellido (3)}
\pertenenciaJurTres{pertenencia (3)}
 
\fechaINICIO{20 de octubre de 2022}		%Fecha de inicio de la cursada de GdP \fechaInicioName
\fechaFINALPlan{8 de diciembre de 2022} 	%Fecha de final de cursada de GdP
\fechaFINALTrabajo{8 de diciembre de 2023}	%Fecha de defensa pública del trabajo final


\begin{document}

\maketitle
\thispagestyle{empty}
\pagebreak


\thispagestyle{empty}
{\setlength{\parskip}{0pt}
\tableofcontents{}
}
\pagebreak


\section*{Registros de cambios}
\label{sec:registro}


\begin{table}[ht]
\label{tab:registro}
\centering
\begin{tabularx}{\linewidth}{@{}|c|X|c|@{}}
\hline
\rowcolor[HTML]{C0C0C0} 
Revisión & \multicolumn{1}{c|}{\cellcolor[HTML]{C0C0C0}Detalles de los cambios realizados} & Fecha      \\ \hline
0      & Creación del documento                                 &\fechaInicioName \\ \hline
%1      & Se completa hasta el punto 4 inclusive                 & dd/mm/aaaa \\ \hline
%2      & Se completa hasta el punto 7 inclusive
%		  Se puede agregar algo más \newline
%		  En distintas líneas \newline
%		  Así                                                    & dd/mm/aaaa \\ \hline
%3      & Se completa hasta el punto 11 inclusive                & dd/mm/aaaa \\ \hline
%4      & Se completa el plan	                                 & dd/mm/aaaa \\ \hline
\end{tabularx}
\end{table}

\pagebreak



\section*{Acta de constitución del proyecto}
\label{sec:acta}

\begin{flushright}
Buenos Aires, \fechaInicioName
\end{flushright}

\vspace{2cm}

Por medio de la presente se acuerda con el Ing. \authorname\hspace{1px} que su Trabajo Final de la \degreename\hspace{1px} se titulará ``\ttitle'', consistirá esencialmente en la implementación de un prototipo de un sistema de control de equipos de rayos X, y tendrá un presupuesto preliminar estimado de 600 hs de trabajo y \$300000, con fecha de inicio \fechaInicioName\hspace{1px} y fecha de presentación pública \fechaFinalName.

Se adjunta a esta acta la planificación inicial.

\vfill

% Esta parte se construye sola con la información que hayan cargado en el preámbulo del documento y no deben modificarla
\begin{table}[ht]
\centering
\begin{tabular}{ccc}
\begin{tabular}[c]{@{}c@{}}Ariel Lutenberg \\ Director posgrado FIUBA\end{tabular} & \hspace{2cm} & \begin{tabular}[c]{@{}c@{}}\clientename \\ \empclientename \end{tabular} \vspace{2.5cm} \\ 
\multicolumn{3}{c}{\begin{tabular}[c]{@{}c@{}} \supname \\ Director del Trabajo Final\end{tabular}} \vspace{2.5cm} \\
%\begin{tabular}[c]{@{}c@{}}\jurunoname \\ Jurado del Trabajo Final\end{tabular}     &  & \begin{tabular}[c]{@{}c@{}}\jurdosname\\ Jurado del Trabajo Final\end{tabular}  \vspace{2.5cm}  \\
%\multicolumn{3}{c}{\begin{tabular}[c]{@{}c@{}} \jurtresname\\ Jurado del Trabajo Final\end{tabular}} \vspace{.5cm}                                                                     
\end{tabular}
\end{table}


\section{0. Definiciones, Acrónimos y Abreviaturas}
\label{sec:definiciones}
\begin{tabular}{l|l}
	Disparo 		& Exposición controlada de RX \\
	Empresa		& SIMEN Rx, dedicada a los equipos de RX, cliente del software \\
	F 			& Frecuencia de trabajo del equipo de RX (excitación del TFRX) \\
	Filamento	& Filamento dentro del Tubo de RX, el cual emite los electrones \\
	FS 			& Fluoro Scopia, exposición  de mayor duración y menor potencia \\
	HF 			& Alta frecuencia, en equipos modernos, F > 14kHz \\
	kV 			& Tensión de trabajo en el secundario del TFRX \\
	LF 			& Baja frecuencia, en equipos convencionales, F < 400Hz \\
	Línea		& Tensión de alimentación del equipo de RX (típica 220V, 50Hz) \\
	mA			& Corriente de filamento en el primario del TFRX \\
	N/A			& No aplica \\
	Preparación	& Etapa previa al Disparo, en la cual se enciende el Filamento \\
	PWM			& Modulación por ancho de pulsos \\
	RX			& Rayos X \\
	SCRX			& Software de Control de Rayos X \\
	Técnicos		& Técnicos radiólogos, usuarios del software \\
	TFRX			& Transformador de RX, eleva la tensión a aplicar en el Tubo de RX \\
	Tubo de RX	& Ampolla en vacío dentro de la cual se generan los RX \\
\end{tabular} 



\section{1. Descripción técnica-conceptual del proyecto a realizar}
\label{sec:descripcion}

El proyecto consiste en el diseño, construcción y puesta en marcha de un prototipo de sistema de control genérico de equipos de rayos X, para la Empresa SIMEN Rx, en adelante la Empresa. El software a desarrollar llevará el nombre comercial de SCRX (Software de Control para equipos de Rayos X).

La Empresa, radicada en el departamento de Godoy Cruz de la ciudad de Mendoza y fundada en 1993, cumplirá el rol de cliente del software en cuestión. A su vez, la Empresa tiene como sus propios clientes a los dueños de hospitales, clínicas, veterinarias, particulares, etc. En todos los casos, los que finalmente manejan los equipos son los Técnicos radiólogos a cargo, en adelante los Técnicos, los cuales cumplirán el rol de usuarios del software en cuestión.

La Empresa se dedica a los equipos hospitalarios, con especialización en rayos X. Actualmente cuenta con servicios de venta, reparación, traslado, instalación, servicio técnico y mantenimiento preventivo y correctivo de equipos de rayos X y es líder del mercado en la zona Cuyo del país. Utiliza métodos de electrónica general y de potencia en equipos convencionales (de baja frecuencia, en adelante LF) y modernos (de alta frecuencia, en adelante HF), además de software y programación de microcontroladores para la etapa de control. Otros procesos que realiza la Empresa están relacionados con el mecanizado de precisión, torneado, fresado, soldadura, corte, etc. También realiza montaje mecánico en general.

La Empresa requiere de un sistema embebido que permita la selección y control de todas las variables involucradas en un estudio de radioterapia, tales como tensión de alimentación, tensión de trabajo del equipo, corriente de filamento, tiempo de Disparo, selección de foco, control de Preparación y Disparo, entre otras. El sistema a desarrollar debe cumplir con determinadas pautas de seguridad y debe ser capaz de adaptarse a equipos tanto convencionales como modernos, así como también a situaciones particulares que solicite cada cliente de la Empresa. Deberá además mostrar toda la información necesaria para el correcto uso del equipo a los Técnicos, mediante un display y/o algún tipo de interfaz gráfica en computadora/celular.

% La regla es que las figuras nunca pueden ir antes de ser mencionadas en el texto, porque sino el lector no entiende por qué de pronto aparece una figura.

%Los colores usados en el diagrama deben ser adecuados, tal que ayuden a comprender mejor el diagrama, preferentemente en la gama de colores pastel.

En la Figura \ref{fig:diagBloques} se presenta el diagrama en bloques del sistema. 

\begin{figure}[htpb]
\centering 
\includegraphics[width=.6\textwidth]{./Figuras/diagBloques.png}
\caption{Diagrama en bloques del sistema}
\label{fig:diagBloques}
\end{figure}
\vspace{25px}

Se observa que el sistema será capaz de leer variables físicas del equipo mediante sus entradas analógicas. Placas electrónicas externas se encargan de traducir dichas variables a tensiones en el rango de 0-Vcc, compatibles con el microcontrolador. Por otro lado, leerá entradas tanto de pulsadores (Preparación, Disparo, etc.) como de un teclado matricial, así como también permitirá la activación de etapas de potencia, ya sea con señales lógicas ON/OFF o de Modulación de Ancho de pulsos, con estrategias de control acorde a la aplicación. Se podrá comunicar con una computadora de control  u otros dispositivos mediante la interfaz serie asíncrona UART. Por último, mostrará toda la información en un Display alfanumérico para interactuar con el usuario.

El desafío es brindar a cada cliente de la Empresa la posibilidad de adquirir un equipo nuevo o actualizar el que ya posee, con tecnología moderna, confiable, segura y duradera, brindando comodidad y practicidad en su uso, permitiendo ajustar las variables del equipo para maximizar su rendimiento y calidad en la imagen radiográfica final. Además, se busca que esta solución sea de un costo razonable,  para incrementar la posibilidad de su adquisición a todo tipo de clientes, tales como veterinarios y radiólogos independientes. 

El camino hacia la consecución de este objetivo ya comenzó hace dos años, resolviendo problemas modularmente mediante sistemas embebidos con microcontroladores PIC y AVR, y otros sistemas electrónicos interconectados entre sí. Se quiere dar un salto tecnológico a microcontroladores más potentes, que permitan lograr mayor alcance y funcionalidades. Se busca también vincular los sistemas, antes independientes, minimizando espacios físicos, costos, tiempos de instalación y fallas posibles; y maximizando la posibilidad de incorporar nuevas funcionalidades en el futuro.
 
%El objetivo es que el lector en una o dos páginas entienda de qué se trata el proyecto y cuáles son sus desafíos, su motivación y su importancia.

%Se debe destacar claramente cuál es el valor que agrega el proyecto a realizar. ``El presente proyecto se destaca especialmente por incorporar tal cosa... Esto lo diferencia de otros sistemas similares en que ...''

%Puede ser útil incluir en esta sección la respuesta a alguna de estas preguntas:
%¿Cómo se vincula este proyecto con la misión de la organización?
%¿Cómo se inserta este proyecto en el modelo de negocio de la organización?
%¿Ayuda a la explicación si se incluye un lienzo Canvas del Modelo de Negocio?
%¿En qué estado del ciclo de vida está el producto que se desea reemplazar o mejorar?
%¿Cuáles son las necesidades que debe satisfacer?
%¿Por dónde pasa la innovación?

\section{2. Identificación y análisis de los interesados}
\label{sec:interesados}

% Es inusual que una misma persona esté en más de un rol, incluso en proyectos chicos. 
% Si se considera que una persona cumple dos o más roles, entonces sólo dejarla en el rol más importante. Por ejemplo:
%\begin{itemize}
%\item Si una persona es Cliente pero también colabora u orienta, dejarla solo como Cliente.
%\item Si una persona es el Responsable, no debe ser colocado también como Miembro del equipo.
%\end{itemize}
% Pero en cambio sí es usual que el Cliente y el Auspiciante sean el mismo, por ejemplo.

\begin{table}[ht]
%\caption{Identificación de los interesados}
%\label{tab:interesados}
\begin{tabularx}{\linewidth}{@{}|l|X|X|l|@{}}
\hline
\rowcolor[HTML]{C0C0C0} 
Rol           	& Nombre y Apellido & Organización 		& Puesto 					\\ \hline
Auspiciante   	& \clientename      & \empclientename  	& Director Administrativo    \\ \hline
Cliente       	& \clientename      & \empclientename	& Director Administrativo    \\ \hline
Responsable   	& \authorname       & FIUBA        		& Alumno 					\\ \hline
Colaboradores 	& Cristian Vatri    & ECAVSA   			& Circuitos Impresos			\\ \newline
				& Alejo Vila        & Kraneal 			& Impresión 3D  				\\ \hline
Orientador    	& \supname	        & \pertesupname 		& Director Trabajo final 	\\ \hline
Usuario final 	& Daniel Daza       & \empclientename	& Técnico 					\\ \hline
\end{tabularx}
\end{table}

%El Director suele ser uno de los Orientadores.
%No dejar celdas vacías; si no hay nada que poner en una celda colocar un signo ``-''.
%No dejar filas vacías; si no hay nada que poner en una fila entonces eliminarla.

A continuación se listan las principales características de cada interesado.
 
\begin{itemize}
	\item Auspiciante: siempre busca agregar valor al Usuario. Estar atento a sus recomendaciones.
	\item Colaboradores: es difícil estimar los tiempos de sus trabajos, muchas veces porque algunas demoras no dependen en forma directa de ellos. Planificar con tiempo y consultar a menudo su avance.
	\item Orientador: está siempre muy ocupado, organizar y optimizar las consultas.
	\item Usuario final: intenta todas las formas posibles de hacer fallar el equipo hasta que lo logra. Si pasa su prueba, el sistema es muy robusto.
\end{itemize}


\section{3. Propósito del proyecto}
\label{sec:proposito}

El propósito de este proyecto es brindar, tanto a los clientes existentes de la Empresa como a nuevos clientes, la posibilidad de adquirir un equipo nuevo o actualizar el que ya poseen, agregando valor a su trabajo diario. Será prioridad que esta solución sea de un costo razonable, para incrementar la posibilidad de su adquisición a mayor cantidad de clientes. 

\section{4. Alcance del proyecto}
\label{sec:alcance}

El presente proyecto incluye el diseño, construcción y prueba de funcionamiento de un prototipo basado en sistema embebido para el control genérico de equipos de rayos X, capaz de interactuar con el Técnico para que éste pueda configurar los parámetros necesarios del equipo y observar la información en un display o interfaz. 

El presente proyecto no incluye la comercialización del sistema, su inclusión dentro de cada equipo ni la creación de la interfaz mecánica entre el equipo y el sistema. Tampoco se proveen las fuentes de alimentación requeridas, así como toda instalación eléctrica. 

\section{5. Supuestos del proyecto}
\label{sec:supuestos}

Para el desarrollo del presente proyecto se supone que:

\begin{itemize}
	\item Se tendrá disponibilidad y accesibilidad a cada equipo de RX al que se desee introducir el SCRX, para interactuar durante el desarrollo del software, con la posibilidad de realizar pruebas y calibración. 
	\item Se tendrán los instrumentos necesarios para la medición de todas las variables necesarias del sistema, a  saber: osciloscopio digital de dos canales y 1 MHz de velocidad; multímetro con capacidad de lectura de tensiones alternas hasta 700 Vca, tensiones continuas hasta 500 Vcc, corrientes desde 1 uA hasta 10 A, continuidad, juntura de diodos, capacidad hasta 10 uF; inductímetro con capacidad de lectura de inductancias desde 1 uH hasta 10 H, dosímetro para detección y medición de radiación ionizante.
	\item Se tomarán los recaudos necesarios para no exponerse a la radiación en ninguna circunstancia, a saber: chalecos, vidrios y cajones plomados en buenas condiciones.
\end{itemize}

%Por ejemplo, se podrían incluir supuestos respecto a disponibilidad de tiempo y recursos humanos y materiales, sobre la factibilidad técnica de distintos aspectos del proyecto, sobre otras cuestiones que sean necesarias para el éxito del proyecto como condiciones macroeconómicas o reglamentarias.

\section{6. Requerimientos}
\label{sec:requerimientos}

\begin{consigna}{red}
Los requerimientos deben numerarse y de ser posible estar agruparlos por afinidad, por ejemplo:

\begin{enumerate}
	\item Requerimientos funcionales
		\begin{enumerate}
			\item El sistema debe...
			\item Tal componente debe...
			\item El usuario debe poder...
		\end{enumerate}
	\item Requerimientos de documentación
		\begin{enumerate}
			\item Requerimiento 1
			\item Requerimiento 2 (prioridad menor)
		\end{enumerate}
	\item Requerimiento de testing...
	\item Requerimientos de la interfaz...
	\item Requerimientos interoperabilidad...
	\item etc...
\end{enumerate}

Leyendo los requerimientos se debe poder interpretar cómo será el proyecto y su funcionalidad.

Indicar claramente cuál es la prioridad entre los distintos requerimientos y si hay requerimientos opcionales. 

No olvidarse de que los requerimientos incluyen a las regulaciones y normas vigentes!!!

Y al escribirlos seguir las siguientes reglas:
\begin{itemize}
	\item Ser breve y conciso (nadie lee cosas largas). 
	\item Ser específico: no dejar lugar a confusiones.
	\item Expresar los requerimientos en términos que sean cuantificables y medibles.
\end{itemize}

\end{consigna}

\section{7. Historias de usuarios (\textit{Product backlog})}
\label{sec:backlog}

\begin{consigna}{red}
Descripción: En esta sección se deben incluir las historias de usuarios y su ponderación (\textit{history points}). Recordar que las historias de usuarios son descripciones cortas y simples de una característica contada desde la perspectiva de la persona que desea la nueva capacidad, generalmente un usuario o cliente del sistema. La ponderación es un número entero que representa el tamaño de la historia comparada con otras historias de similar tipo.

El formato propuesto es: "como [rol] quiero [tal cosa] para [tal otra cosa]."

Se debe indicar explícitamente el criterio para calcular los \textit{story points} de cada historia
\end{consigna}

\section{8. Entregables principales del proyecto}
\label{sec:entregables}

\begin{consigna}{red}

Los entregables del proyecto son (ejemplo):

\begin{itemize}
	\item Manual de uso
	\item Diagrama de circuitos esquemáticos
	\item Código fuente del firmware
	\item Diagrama de instalación
	\item Informe final
	\item etc...
\end{itemize}

\end{consigna}

\section{9. Desglose del trabajo en tareas}
\label{sec:wbs}

\begin{consigna}{red}
El WBS debe tener relación directa o indirecta con los requerimientos.  Son todas las actividades que se harán en el proyecto para dar cumplimiento a los requerimientos. Se recomienda mostrar el WBS mediante una lista indexada:

\begin{enumerate}
\item Grupo de tareas 1
	\begin{enumerate}
	\item Tarea 1 (tantas hs)
	\item Tarea 2 (tantas hs)
	\item Tarea 3 (tantas hs)
	\end{enumerate}
\item Grupo de tareas 2
	\begin{enumerate}
	\item Tarea 1 (tantas hs)
	\item Tarea 2 (tantas hs)
	\item Tarea 3 (tantas hs)
	\end{enumerate}
\item Grupo de tareas 3
	\begin{enumerate}
	\item Tarea 1 (tantas hs)
	\item Tarea 2 (tantas hs)
	\item Tarea 3 (tantas hs)
	\item Tarea 4 (tantas hs)
	\item Tarea 5 (tantas hs)
	\end{enumerate}
\end{enumerate}

Cantidad total de horas: (tantas hs)

Se recomienda que no haya ninguna tarea que lleve más de 40 hs. 

\end{consigna}

\section{10. Diagrama de Activity On Node}
\label{sec:AoN}

\begin{consigna}{red}
Armar el AoN a partir del WBS definido en la etapa anterior. 

%La figura \ref{fig:AoN} fue elaborada con el paquete latex tikz y pueden consultar la siguiente referencia \textit{online}:

%\url{https://www.overleaf.com/learn/latex/LaTeX_Graphics_using_TikZ:_A_Tutorial_for_Beginners_(Part_3)\%E2\%80\%94Creating_Flowcharts}

\end{consigna}

\begin{figure}[htpb]
\centering 
\includegraphics[width=.8\textwidth]{./Figuras/AoN.png}
\caption{Diagrama en \textit{Activity on Node}}
\label{fig:AoN}
\end{figure}

Indicar claramente en qué unidades están expresados los tiempos.
De ser necesario indicar los caminos semicríticos y analizar sus tiempos mediante un cuadro.
Es recomendable usar colores y un cuadro indicativo describiendo qué representa cada color, como se muestra en el siguiente ejemplo:



\section{11. Diagrama de Gantt}
\label{sec:gantt}

\begin{consigna}{red}

Existen muchos programas y recursos \textit{online} para hacer diagramas de gantt, entre los cuales destacamos:

\begin{itemize}
\item Planner
\item GanttProject
\item Trello + \textit{plugins}. En el siguiente link hay un tutorial oficial: \\ \url{https://blog.trello.com/es/diagrama-de-gantt-de-un-proyecto}
\item Creately, herramienta online colaborativa. \\\url{https://creately.com/diagram/example/ieb3p3ml/LaTeX}
\item Se puede hacer en latex con el paquete \textit{pgfgantt}\\ \url{http://ctan.dcc.uchile.cl/graphics/pgf/contrib/pgfgantt/pgfgantt.pdf}
\end{itemize}

Pegar acá una captura de pantalla del diagrama de Gantt, cuidando que la letra sea suficientemente grande como para ser legible. 
Si el diagrama queda demasiado ancho, se puede pegar primero la ``tabla'' del Gantt y luego pegar la parte del diagrama de barras del diagrama de Gantt.

Configurar el software para que en la parte de la tabla muestre los códigos del EDT (WBS).\\
Configurar el software para que al lado de cada barra muestre el nombre de cada tarea.\\
Revisar que la fecha de finalización coincida con lo indicado en el Acta Constitutiva.

En la figura \ref{fig:gantt}, se muestra un ejemplo de diagrama de gantt realizado con el paquete de \textit{pgfgantt}. En la plantilla pueden ver el código que lo genera y usarlo de base para construir el propio.

\begin{figure}[htbp]
\begin{center}
\begin{ganttchart}{1}{12}
  \gantttitle{2020}{12} \\
  \gantttitlelist{1,...,12}{1} \\
  \ganttgroup{Group 1}{1}{7} \\
  \ganttbar{Task 1}{1}{2} \\
  \ganttlinkedbar{Task 2}{3}{7} \ganttnewline
  \ganttmilestone{Milestone o hito}{7} \ganttnewline
  \ganttbar{Final Task}{8}{12}
  \ganttlink{elem2}{elem3}
  \ganttlink{elem3}{elem4}
\end{ganttchart}
\end{center}
\caption{Diagrama de gantt de ejemplo}
\label{fig:gantt}
\end{figure}


\begin{landscape}
\begin{figure}[htpb]
\centering 
\includegraphics[height=.85\textheight]{./Figuras/Gantt-2.png}
\caption{Ejemplo de diagrama de Gantt rotado}
\label{fig:diagGantt}
\end{figure}

\end{landscape}

\end{consigna}


\section{12. Presupuesto detallado del proyecto}
\label{sec:presupuesto}

\begin{consigna}{red}
Si el proyecto es complejo entonces separarlo en partes:
\begin{itemize}
	\item Un total global, indicando el subtotal acumulado por cada una de las áreas.
	\item El desglose detallado del subtotal de cada una de las áreas.
\end{itemize}

IMPORTANTE: No olvidarse de considerar los COSTOS INDIRECTOS.

\end{consigna}

\begin{table}[htpb]
\centering
\begin{tabularx}{\linewidth}{@{}|X|c|r|r|@{}}
\hline
\rowcolor[HTML]{C0C0C0} 
\multicolumn{4}{|c|}{\cellcolor[HTML]{C0C0C0}COSTOS DIRECTOS} \\ \hline
\rowcolor[HTML]{C0C0C0} 
Descripción &
  \multicolumn{1}{c|}{\cellcolor[HTML]{C0C0C0}Cantidad} &
  \multicolumn{1}{c|}{\cellcolor[HTML]{C0C0C0}Valor unitario} &
  \multicolumn{1}{c|}{\cellcolor[HTML]{C0C0C0}Valor total} \\ \hline
 &
  \multicolumn{1}{c|}{} &
  \multicolumn{1}{c|}{} &
  \multicolumn{1}{c|}{} \\ \hline
 &
  \multicolumn{1}{c|}{} &
  \multicolumn{1}{c|}{} &
  \multicolumn{1}{c|}{} \\ \hline
\multicolumn{1}{|l|}{} &
   &
   &
   \\ \hline
\multicolumn{1}{|l|}{} &
   &
   &
   \\ \hline
\multicolumn{3}{|c|}{SUBTOTAL} &
  \multicolumn{1}{c|}{} \\ \hline
\rowcolor[HTML]{C0C0C0} 
\multicolumn{4}{|c|}{\cellcolor[HTML]{C0C0C0}COSTOS INDIRECTOS} \\ \hline
\rowcolor[HTML]{C0C0C0} 
Descripción &
  \multicolumn{1}{c|}{\cellcolor[HTML]{C0C0C0}Cantidad} &
  \multicolumn{1}{c|}{\cellcolor[HTML]{C0C0C0}Valor unitario} &
  \multicolumn{1}{c|}{\cellcolor[HTML]{C0C0C0}Valor total} \\ \hline
\multicolumn{1}{|l|}{} &
   &
   &
   \\ \hline
\multicolumn{1}{|l|}{} &
   &
   &
   \\ \hline
\multicolumn{1}{|l|}{} &
   &
   &
   \\ \hline
\multicolumn{3}{|c|}{SUBTOTAL} &
  \multicolumn{1}{c|}{} \\ \hline
\rowcolor[HTML]{C0C0C0}
\multicolumn{3}{|c|}{TOTAL} &
   \\ \hline
\end{tabularx}%
\end{table}


\section{13. Gestión de riesgos}
\label{sec:riesgos}

\begin{consigna}{red}
a) Identificación de los riesgos (al menos cinco) y estimación de sus consecuencias:
 
Riesgo 1: detallar el riesgo (riesgo es algo que si ocurre altera los planes previstos de forma negativa)
\begin{itemize}
	\item Severidad (S): mientras más severo, más alto es el número (usar números del 1 al 10).\\
	Justificar el motivo por el cual se asigna determinado número de severidad (S).
	\item Probabilidad de ocurrencia (O): mientras más probable, más alto es el número (usar del 1 al 10).\\
	Justificar el motivo por el cual se asigna determinado número de (O). 
\end{itemize}   

Riesgo 2:
\begin{itemize}
	\item Severidad (S): 
	\item Ocurrencia (O):
\end{itemize}

Riesgo 3:
\begin{itemize}
	\item Severidad (S): 
	\item Ocurrencia (O):
\end{itemize}


b) Tabla de gestión de riesgos:      (El RPN se calcula como RPN=SxO)

\begin{table}[htpb]
\centering
\begin{tabularx}{\linewidth}{@{}|X|c|c|c|c|c|c|@{}}
\hline
\rowcolor[HTML]{C0C0C0} 
Riesgo & S & O & RPN & S* & O* & RPN* \\ \hline
       &   &   &     &    &    &      \\ \hline
       &   &   &     &    &    &      \\ \hline
       &   &   &     &    &    &      \\ \hline
       &   &   &     &    &    &      \\ \hline
       &   &   &     &    &    &      \\ \hline
\end{tabularx}%
\end{table}

Criterio adoptado: 
Se tomarán medidas de mitigación en los riesgos cuyos números de RPN sean mayores a...

Nota: los valores marcados con (*) en la tabla corresponden luego de haber aplicado la mitigación.

c) Plan de mitigación de los riesgos que originalmente excedían el RPN máximo establecido:
 
Riesgo 1: plan de mitigación (si por el RPN fuera necesario elaborar un plan de mitigación).
  Nueva asignación de S y O, con su respectiva justificación:
  - Severidad (S): mientras más severo, más alto es el número (usar números del 1 al 10).
          Justificar el motivo por el cual se asigna determinado número de severidad (S).
  - Probabilidad de ocurrencia (O): mientras más probable, más alto es el número (usar del 1 al 10).
          Justificar el motivo por el cual se asigna determinado número de (O).

Riesgo 2: plan de mitigación (si por el RPN fuera necesario elaborar un plan de mitigación).
 
Riesgo 3: plan de mitigación (si por el RPN fuera necesario elaborar un plan de mitigación).

\end{consigna}


\section{14. Gestión de la calidad}
\label{sec:calidad}

\begin{consigna}{red}
Para cada uno de los requerimientos del proyecto indique:
\begin{itemize} 
\item Req \#1: copiar acá el requerimiento.

\begin{itemize}
	\item Verificación para confirmar si se cumplió con lo requerido antes de mostrar el sistema al cliente. Detallar 
	\item Validación con el cliente para confirmar que está de acuerdo en que se cumplió con lo requerido. Detallar  
\end{itemize}

\end{itemize}

Tener en cuenta que en este contexto se pueden mencionar simulaciones, cálculos, revisión de hojas de datos, consulta con expertos, mediciones, etc.  Las acciones de verificación suelen considerar al entregable como ``caja blanca'', es decir se conoce en profundidad su funcionamiento interno.  En cambio, las acciones de validación suelen considerar al entregable como ``caja negra'', es decir, que no se conocen los detalles de su funcionamiento interno.

\end{consigna}

\section{15. Procesos de cierre}    
\label{sec:cierre}

\begin{consigna}{red}
Establecer las pautas de trabajo para realizar una reunión final de evaluación del proyecto, tal que contemple las siguientes actividades:

\begin{itemize}
	\item Pautas de trabajo que se seguirán para analizar si se respetó el Plan de Proyecto original:
	 - Indicar quién se ocupará de hacer esto y cuál será el procedimiento a aplicar. 
	\item Identificación de las técnicas y procedimientos útiles e inútiles que se emplearon, y los problemas que surgieron y cómo se solucionaron:
	 - Indicar quién se ocupará de hacer esto y cuál será el procedimiento para dejar registro.
	\item Indicar quién organizará el acto de agradecimiento a todos los interesados, y en especial al equipo de trabajo y colaboradores:
	  - Indicar esto y quién financiará los gastos correspondientes.
\end{itemize}

\end{consigna}


\end{document}
